%%%%%%%%%%%%%%%%%%%%%%%%%%%%%%%%%%%%%%%%%%%%%
%% Conclusão e Trabalhos Futuros
%% Copyright 2006 Eliézio Batista de Oliveira
%%%%%%%%%%%%%%%%%%%%%%%%%%%%%%%%%%%%%%%%%%%%%

\chapter{Conclusão e Trabalhos Futuros}

\section{Conclusão}

Apesar da RFC~3546 ter sido publicada há quase três anos, somente agora
tem surgido implementações das extensões propostas, mas mesmo assim parciais.

A versão em desenvolvimento do OpenSSL (0.9.9) já contém uma implementação funcional da
extensão \acl{SNI} que também é contemplada na vindoura versão do \emph{browser} Opera (9.0)
e na pilha GnuTLS.

As demais extensões, todas voltadas para sistemas embutidos, tem recebido pouca ou quase nenhuma
atenção dos desenvolvedores de \emph{software} \emph{open-source}, assim como de empresas comerciais.

Com base na minha experiência em desenvolvimento de soluções de comunicação, entendo que esse interesse
tem sido atenuado por alguns mitos:

\begin{itemize}

\item De que o ostensivo e contínuo aumento na banda em praticamente todos os meios de comunicação dispensaria a necessidade de esforços para a otimização de protocolos. Essa expectativa não procede pois o aumento na demanda tem sido ainda mais vertiginoso, principalmente em redes utilizadas por equipamentos móveis;

\item Que o mecanismo de restauração de sessões TLS anularia a aplicabilidade da maioria destas
extensões. De fato, com excessão das extensões \acl{MFL} e \acl{TMAC}, todas as outras cinco só são
aplicáveis em \emph{handshakes} completos.

No entanto, ao contrário do muitos supõem, em geral as sessões TLS não são mantidas por muito tempo pelos servidores HTTPS -- os maiores beneficiados por este mecanismo -- por causa do espaço em memória que seria necessário para armazená-las em larga escala.
No site \emph{Amazon.com}, por exemplo, as sessões TLS são descartadas após breves 2 minutos!

\end{itemize}

A minha real expectativa é de que a implementação alvo deste projeto possa contribuir efetivamente
para popularizar o uso dessas extensões para um uso mais racional dos recursos disponíveis para
sistemas restritos.

\section{Propostas de Trabalhos Futuros}

\subsection{Tarefas Gerais}

\begin{itemize}

\item Submissão ao \emph{core team} de desenvolvedores do OpenSSL a fim de 
oficializar esta implementação;

\item Replicar a API proposta no nível \texttt{CTX}, visando a consistência com a API 
oficial do OpenSSL;

\item Garantir que as novas implementações são \emph{thread-safe};

\item Incorporação de outras possíveis extensões já definidas pelo \emph{\acs{IETF} TLS 
Working Group} ou que ainda estejam em fase de discussão. Exemplo: a que 
permite a substituição do \acs{SHA1} por algum algoritmo de \emph{hashing} mais 
seguro (SHA-256 ou SHA-512, por exemplo);

\item Implementar estas extensões em uma pilha SSL/TLS voltada para
sistemas embutidos. O candidato mais adequado parece ser o MatrixSSL (\url{http://www.matrixssl.org/}),
principalmente por causa do seu duplo-licenciamento (GPL e comercial);

\item Corrigir o Ethereal para decodificar corretamente a mensagem \tlsHsCu
e ser capaz de desfragmentar registros;

\end{itemize}

\subsection{Extensão \acl{SNI}}

\begin{itemize}

\item Implementar a parte servidora, incluindo as necessárias modificações
no \emph{``mod\_ssl''} (\url{http://httpd.apache.org/docs/2.0/ssl/}) para ativar e configurar os
seus diversos parâmetros.

\end{itemize}

\subsection{Extensão \acl{CCU}}

\begin{itemize}

\item Implementar o \emph{cache} de certificados no servidor;

\item Analisar a possibilidade de usar algum esquema específico em que o
cliente informe somente o contexto (ou \emph{domain}/\emph{realm}) e um
identificador (\emph{serial number}/\emph{username}) e que o servidor seja
capaz de ``montar'' uma \acs{URL} com base nessas informações e em \emph{template}
configurado;

\item Implementar mecanismos que restrinjam o acesso a um conjunto limitado
de provedores de certificados. Visar também minimizar eventuais ataques do
tipo \ac{DoS}.

\end{itemize}

\subsection{Extensão \acl{CSR}}

\begin{itemize}

\item Além de implementar esta extensão, avaliar a possibilidade do uso de
outros protocolos alternativos, tais como \ac{SCVP}, \ac{DCS} e \ac{OCSPX};

\end{itemize}

\subsection{Extensão \acl{SCCI}}

\begin{itemize}

\item Obtenção de um identificador oficial para esta extensão junto a
\acs{IANA}/\acs{IETF} (vide \url{http://www.iana.org/assignments/tls-extensiontype-values});

\item Submissão desta nova extensão ao grupo de trabalho do \acs{IETF}
responsável pelo TLS para uma possível criação de uma \emph{Internet-Draft} e
futuramente de uma RFC;

\end{itemize}

