%%%%%%%%%%%%%%%%%%%%%%%%%%%%%%%%%%%%%%%%%%%%%
%% Capítulo 2 - Client Certificate URL
%% Copyright 2006 Eliézio Batista de Oliveira
%%%%%%%%%%%%%%%%%%%%%%%%%%%%%%%%%%%%%%%%%%%%%

\section{\acl{CCU}}

\subsection{Especificação}

\begin{lstlisting}[caption={Definição do novo tipo de mensagem de \emph{handshake} \tlsHsCu}]
   enum {
       ..., certificate_url(21), ...,
       (255)
   } HandshakeType;
\end{lstlisting}

\begin{lstlisting}[caption={RFC 3546, trecho da seção 3.3}]
After negotiation of the use of client certificate URLs has been
successfully completed (by exchanging hellos including
"client_certificate_url" extensions), clients MAY send a
"CertificateURL" message in place of a "Certificate" message:

   struct {
       CertChainType type;
       URLAndOptionalHash url_and_hash_list<1..2^16-1>;
   } CertificateURL;

   enum {
       individual_certs(0), pkipath(1), (255)
   } CertChainType;

   struct {
       opaque url<1..2^16-1>;
       Boolean hash_present;
       select (hash_present) {
           case false: struct {};
           case true: SHA1Hash;
       } hash;
   } URLAndOptionalHash;

   enum {
       false(0), true(1)
   } Boolean;

   opaque SHA1Hash[20];

Here "url_and_hash_list" contains a sequence of URLs and optional
hashes.

When X.509 certificates are used, there are two possibilities:

-  if CertificateURL.type is "individual_certs", each URL refers to
   a single DER-encoded X.509v3 certificate, with the URL for the
   client's certificate first, or

-  if CertificateURL.type is "pkipath", the list contains a single
   URL referring to a DER-encoded certificate chain, using the type
   PkiPath described in Section 8.
\end{lstlisting}

\subsection{Divergências}

\begin{enumerate}
\item Os certificados individuais podem estar tanto no formato \acs{PEM} quanto
\acs{DER} (oficial). Foi implementada uma heurística simples para identificar
o formato;

\item A aplicação servidora não está exigindo que o tipo \acs{MIME} indicado na
resposta HTTP seja \verb|application/pkix-pkipath|, como dita a
especificação.
\end{enumerate}

\subsection{API Estendida}

\begin{lstlisting}[language=C,%
		    emph={tlsx_cert_chain_empty,tlsx_cert_chain_add,%
		    TLSX_set_certificate_chain,TLSX_CTX_set_certificate_fetcher_callback},%
		    caption={API Estendida para a extensão \acs{CCU}}]
/* ----- Client Side ----- */
void tlsx_cert_chain_empty (
    CERTIFICATE_CHAIN *cert_chain
);

int tlsx_cert_chain_add (
    CERTIFICATE_CHAIN *cert_chain,
    int cert_type,    /* TLSX_CERT_CHAIN_TYPE_INDIVIDUAL or
                         TLSX_CERT_CHAIN_TYPE_PKIPATH */
    const char *url,
    const unsigned char *sha1hash  /* optional (may be NULL) */
);

void TLSX_set_certificate_chain (
    SSL *ssl,
    const CERTIFICATE_CHAIN *cert_chain
);

/* ----- Server Side ----- */
void TLSX_CTX_set_certificate_fetcher_callback (
    SSL_CTX *ctx,
    int (*cb) (
        SSL *ssl,
        void *uarg,
        const char *url,
        unsigned char **data,
        unsigned int *data_len
    ),
    void *uarg
);
\end{lstlisting}

\subsection{Implementação}

Como a aplicação servidora, ao aceitar essa extensão, assume a
responsabilidade de obter a cadeia de certificados do cliente, fez-se necessário
embutir esse mecanismo na aplicação \verb|s_server|. Optou-se pela criação de um
novo módulo de nome \textit{``wget.c''} que, através da biblioteca \emph{open source} \emph{libcurl}
\cite{Libcurl} implementa um cliente \acs{HTTP} genérico. O pacote de
desenvolvimento desta biblioteca (\verb|libcurl3-dev|, versão 7.14.0, no ambiente Linux de
testes) deve estar previamente instalado na máquina para \emph{build} da aplicação
\verb|s_server|.

A geração dos certificados individuais em formato \acs{DER} (\acs{ASN1} binário) a partir
do formato \acs{PEM} (ASCII Base64) é trivialmente realizada através do comando:

\begin{lstlisting}[caption={Conversão de certificados para o formato \acs{DER}}]
openssl x509 -in cert.pem -out cert.der -outform DER
\end{lstlisting}

Como o formato denominado PKIPATH ainda não é nativamente suportado
pelo OpenSSL e como também não se encontrou nenhuma ferramenta genérica
que o suportasse, foi necessária a criação de uma pequena aplicação em Kotlin 1.3
para realizar essa tarefa (ver Apêndice \vref{chap:MkPkiPath}).

A configuração do servidor \acs{HTTP} (Apache) foi modificada para que o tipo
\verb|application/pkix-pkipath| seja retornado para os arquivos com extensão
\texttt{``.pki''}.

\begin{description}[\breaklabel\setlabelstyle{\ttfamily}]

\item[s3\_clnt.c::ssl3\_connect]
	A máquina de estados foi modificada no tratamento do estado
	\verb|SSL3_ST_CW_CERT_A| onde se decide se a nova mensagem \tlsHsCu
	será enviada no lugar da \tlsHsC ou não.

\item[s3\_srvr.c::ssl3\_get\_client\_certificate]
	Passou a aceitar e processar a mensagem \tlsHsCu caso tenha sido
	acordada no estágio inicial do \emph{handshake}. Toda a decodificação dessa nova
	mensagem está embutida nesta função, delegando apenas a recuperação
	dos certificados para as funções \verb|tlsx_retrieve_certificate| e
	\verb|tlsx_retrieve_pkipath| conforme o caso.

\end{description}

\subsection{Testes de Conformidade}

Precisou-se, primeiramente, configurar a aplicação servidora para exibir a
autenticação do cliente, passando a acioná-la como se segue:

\begin{lstlisting}[caption={Chamada da \texttt{s\_server.sh} para forçar a autenticação do cliente}]
./s_server.sh -Verify 2
\end{lstlisting}

A sintaxe usada na aplicação cliente para exercitar essa extensão foi:

\begin{lstlisting}[caption={Chamada da \texttt{s\_client.sh} para teste da extensão \acs{CCU}}]
./s_client.sh -tlsx_cert_url http://localhost/certs/client.crt \
    -key client.key
\end{lstlisting}

Para capturar também o tráfego \acs{HTTP} entre a aplicação \verb|s_server| e o servidor
Apache, a chamada do \emph{tethereal} foi modificada para:

\begin{lstlisting}[caption={Chamada do \texttt{tethereal} para capturar tráfegos TLS e HTTP}]
tethereal -i lo -V -x port 443 or port 80
\end{lstlisting}

\subsubsection{Resultado do teste com certificados individuais}

Novamente esbarramos em uma limitação do \emph{tethereal} que não foi capaz de
decodificar propriamente a nova mensagem \tlsHsCu, que a identificou
erroneamente como \emph{Encrypted Handshake Message}. Mas é possível comprovar a
correção da implementação conferindo diretamente a seqüência de bytes do
registro correspondente a essa nova mensagem, que se encontra destacada na
listagem a seguir:

\begin{lstlisting}[caption={Mensagem \tlsHsCu}]
Secure Socket Layer
  TLS Record Layer: Handshake Protocol: Encrypted Handshake Message
    Content Type: Handshake (22)
    Version: TLS 1.0 (0x0301)
    Length: |43|
    Handshake Protocol: Encrypted Handshake Message
  TLS Record Layer: Handshake Protocol: Client Key Exchange
    Content Type: Handshake (22)
    Version: TLS 1.0 (0x0301)
    Length: 134
    Handshake Protocol: Client Key Exchange
      Handshake Type: Client Key Exchange (16)
      Length: 130
  TLS Record Layer: Handshake Protocol: Certificate Verify
    Content Type: Handshake (22)
    Version: TLS 1.0 (0x0301)
    Length: 134
    Handshake Protocol: Certificate Verify
      Handshake Type: Certificate Verify (15)
      Length: 130
  TLS Record Layer: Change Cipher Spec Protocol: Change Cipher Spec
    Content Type: Change Cipher Spec (20)
    Version: TLS 1.0 (0x0301)
    Length: 1
    Change Cipher Spec Message
  TLS Record Layer: Handshake Protocol: Encrypted Handshake Message
    Content Type: Handshake (22)
    Version: TLS 1.0 (0x0301)
    Length: 36
    Handshake Protocol: Encrypted Handshake Message

0000  00 00 00 00 00 00 00 00 00 00 00 00 08 00 45 00   ..............E.
0010  01 9d ac cf 40 00 40 06 8e 89 7f 00 00 01 7f 00   ....@.@.........
0020  00 01 c7 bb 01 bb fb 40 02 02 fb b0 f8 01 50 18   .......@......P.
0030  7f ff d3 25 00 00 16 03 01 |00 2b 15 00 00 27 00|   |...%......+...'.|
0040  |00 24 00 21 68 74 74 70 3a 2f 2f 6c 6f 63 61 6c|   |.$.!http://local|
0050  |68 6f 73 74 2f 63 65 72 74 73 2f 63 6c 69 65 6e|   |host/certs/clien|
0060  |74 2e 63 72 74 00| 16 03 01 00 86 10 00 00 82 00   |t.crt.|..........
0070  80 82 56 c1 53 4a 26 be f1 23 dd 3a 9d 76 19 8c   ..V.SJ&..#.:.v..
0080  ac 0e d0 e1 72 a3 0d 13 b0 59 cd 35 82 e0 ee 23   ....r....Y.5...#
0090  26 8e d7 61 71 09 1d 46 bd 01 86 a8 8b 57 0f e9   &..aq..F.....W..
                          .......
0180  01 01 16 03 01 00 24 7e c1 60 17 68 94 44 ae dd   ......$~.`.h.D..
0190  1c 90 2c 40 52 28 42 54 4b 25 3c 9b d0 92 68 67   ..,@R(BTK%<...hg
01a0  d1 ca 9e 6c 13 62 0e d3 ed ae 51                  ...l.b....
\end{lstlisting}

\begin{lstlisting}[caption={Solicitação HTTP enviada pela aplicação \texttt{s\_server}}]
Hypertext Transfer Protocol
  GET /certs/client.crt HTTP/1.1\r\n
    Request Method: GET
    Request URI: /certs/client.crt
    Request Version: HTTP/1.1
  Host: localhost\r\n
  Accept: */*\r\n
  \r\n
\end{lstlisting}

\begin{lstlisting}[caption={Início (cabeçalho) da resposta HTTP enviada pelo servidor para a aplicação \texttt{s\_server}}]
Hypertext Transfer Protocol
  HTTP/1.1 200 OK\r\n
    Request Version: HTTP/1.1
    Response Code: 200
  Date: Thu, 05 Jan 2006 17:32:37 GMT\r\n
  Server: Apache/2.0.54 (Ubuntu) DAV/2 SVN/1.2.0 mod_python/3.1.3 Python/2.4.2 PHP/4.4.0-3 mod_ruby/1.2.4 Ruby/1.8.3(2005-06-23)\r\n
  Last-Modified: Thu, 05 Jan 2006 17:01:44 GMT\r\n
  ETag: "240f2-b66-7cbc2e00"\r\n
  Accept-Ranges: bytes\r\n
  Content-Length: 2918\r\n
  Content-Type: application/x-x509-ca-cert\r\n
  \r\n
\end{lstlisting}

\subsubsection{Resultado do teste com PKIPATH}

A sintaxe usada para aplicação cliente foi modificada para:

\begin{lstlisting}
./s_client.sh -tlsx_pkipath_url http://localhost/certs/certs.pki \
    -key client.key
\end{lstlisting}

Valem as mesmas considerações feitas para o teste anterior quanto à análise dos
dados capturados pelo \emph{tethereal}.

\begin{lstlisting}[caption={Mensagem \tlsHsCu}]
Secure Socket Layer
  TLS Record Layer: Handshake Protocol: Encrypted Handshake Message
    Content Type: Handshake (22)
    Version: TLS 1.0 (0x0301)
    Length: |42|
    Handshake Protocol: Encrypted Handshake Message
  TLS Record Layer: Handshake Protocol: Client Key Exchange
    Content Type: Handshake (22)
    Version: TLS 1.0 (0x0301)
    Length: 134
    Handshake Protocol: Client Key Exchange
      Handshake Type: Client Key Exchange (16)
      Length: 130
  TLS Record Layer: Handshake Protocol: Certificate Verify
    Content Type: Handshake (22)
    Version: TLS 1.0 (0x0301)
    Length: 134
    Handshake Protocol: Certificate Verify
      Handshake Type: Certificate Verify (15)
      Length: 130
  TLS Record Layer: Change Cipher Spec Protocol: Change Cipher Spec
    Content Type: Change Cipher Spec (20)
    Version: TLS 1.0 (0x0301)
    Length: 1
    Change Cipher Spec Message
  TLS Record Layer: Handshake Protocol: Encrypted Handshake Message
    Content Type: Handshake (22)
    Version: TLS 1.0 (0x0301)
    Length: 36
    Handshake Protocol: Encrypted Handshake Message

0000  00 00 00 00 00 00 00 00 00 00 00 00 08 00 45 00   ..............E.
0010  01 9c 6a 03 40 00 40 06 d1 56 7f 00 00 01 7f 00   ..j.@.@..V......
0020  00 01 cd 70 01 bb 21 39 b1 cc 20 ef 7e 63 50 18   ...p..!9.. .~cP.
0030  7f ff d2 22 00 00 16 03 01 |00 2a 15 00 00 26 01|   ...".....|.*...&.|
0040  |00 23 00 20 68 74 74 70 3a 2f 2f 6c 6f 63 61 6c|   |.#. http://local|
0050  |68 6f 73 74 2f 63 65 72 74 73 2f 63 65 72 74 73|   |host/certs/certs|
0060  |2e 70 6b 69 00| 16 03 01 00 86 10 00 00 82 00 80   |.pki.|...........
0070  7c fb e4 c7 8a 27 ef 23 2d 18 f8 be f2 97 d1 34   .....'.#-......4
0080  30 b0 a3 4b 1f d8 b5 8e 49 d5 a6 eb 3e ea af 6d   0..K....I...>..m
0090  ae a1 0d 94 cb 66 89 b4 74 d2 86 5d 64 80 c0 96   .....f..t..]d...
                          .......
0180  01 16 03 01 00 24 7e bc d4 51 8f a9 95 33 b6 0c   .....$~..Q...3..
0190  0b c2 db 9b 8d cc 56 28 81 14 a0 74 35 36 3e 8b   ......V(...t56>.
01a0  22 9c f1 f1 0d d3 3e 0f fe 3b                     ".....>..
\end{lstlisting}

\begin{lstlisting}[caption={Solicitação HTTP enviada pela aplicação \texttt{s\_server}}]
Hypertext Transfer Protocol
  GET /certs/certs.pki HTTP/1.1\r\n
    Request Method: GET
    Request URI: /certs/certs.pki
    Request Version: HTTP/1.1
  Host: localhost\r\n
  Accept: */*\r\n
  \r\n
\end{lstlisting}

\begin{lstlisting}[caption={Início (cabeçalho) da resposta HTTP enviada pelo servidor para a aplicação \texttt{s\_server}}]
Hypertext Transfer Protocol
  HTTP/1.1 200 OK\r\n
    Request Version: HTTP/1.1
    Response Code: 200
  Date: Thu, 05 Jan 2006 17:49:30 GMT\r\n
  Server: Apache/2.0.54 (Ubuntu) DAV/2 SVN/1.2.0 mod_python/3.1.3 Python/2.4.2 PHP/4.4.0-3 mod_ruby/1.2.4 Ruby/1.8.3(2005-06-23)\r\n
  Last-Modified: Thu, 05 Jan 2006 16:42:45 GMT\r\n
  ETag: "240f0-654-38d86b40"\r\n
  Accept-Ranges: bytes\r\n
  Content-Length: 1620\r\n
  Content-Type: application/pkix-pkipath\r\n
  \r\n
\end{lstlisting}

