%%%%%%%%%%%%%%%%%%%%%%%%%%%%%%%%%%%%%%%%%%%%%
%% Capítulo 2 - Trusted CA Indication
%% Copyright 2006 Eliézio Batista de Oliveira
%%%%%%%%%%%%%%%%%%%%%%%%%%%%%%%%%%%%%%%%%%%%%

\section{\acl{TCI}}

\subsection{Especificação}

\begin{lstlisting}[caption={RFC 3546, trecho da seção 3.4}]
The "extension_data" field of this extension SHALL contain 
"TrustedAuthorities" where: 
 
   struct { 
       TrustedAuthority trusted_authorities_list<0..2^16-1>; 
   } TrustedAuthorities; 
 
   struct { 
       IdentifierType identifier_type; 
       select (identifier_type) { 
           case pre_agreed: struct {}; 
           case key_sha1_hash: SHA1Hash; 
           case x509_name: DistinguishedName; 
           case cert_sha1_hash: SHA1Hash; 
       } identifier; 
   } TrustedAuthority; 
 
   enum { 
       pre_agreed(0), key_sha1_hash(1), x509_name(2), 
       cert_sha1_hash(3), (255) 
   } IdentifierType; 
 
   opaque DistinguishedName<1..2^16-1>; 
 
 
Here "TrustedAuthorities" provides a list of CA root key identifiers 
that the client possesses.  Each CA root key is identified via 
either: 
 
-  "pre_agreed" - no CA root key identity supplied. 
-  "key_sha1_hash" - contains the SHA-1 hash of the CA root key.  For 
   DSA and ECDSA keys, this is the hash of the "subjectPublicKey" 
   value.  For RSA keys, the hash is of the big-endian byte string 
   representation of the modulus without any initial 0-valued bytes. 
   (This copies the key hash formats deployed in other environments.) 
 
-  "x509_name" - contains the DER-encoded X.509 DistinguishedName of 
   the CA. 
 
-  "cert_sha1_hash" - contains the SHA-1 hash of a DER-encoded 
   Certificate containing the CA root key.
\end{lstlisting}

\subsection{API Estendida}

\begin{lstlisting}[language=C,emph={TLSX_set_trusted_ca},caption={API Estendida para a extensão \acs{TCI}}]
void TLSX_set_trusted_ca (
    SSL *ssl,
    int id_type,    /* TLSX_CERT_ID_PRE_AGREED,
		       TLSX_CERT_ID_KEY_SHA1_HASH,
		       TLSX_CERT_ID_X509_NAME, or
		       TLSX_CERT_ID_CERT_SHA1_HASH */
    const char *CAfile,  /* optional */
    const char *CApath   /* optional */
);
\end{lstlisting}

Essa função associa ao objeto \verb|SSL| uma lista de IDs de todos os certificados 
encontrados no diretório indicado por \verb|CApath| cujo nome base (sem diretório) 
adere ao padrão utilizado pelo OpenSSL para certificados de \acsp{CA}, ou seja, 
segue o formato \texttt{``hhhhhhhh.d''}, onde \texttt{``hhhhhhhh''} é uma seqüência de oito 
dígitos hexadecimais e \texttt{``d''} é um dígito decimal. O certificado opcionalmente 
indicado por \verb|CAfile| é também acrescido a esse conjunto. Todos esses 
certificados devem estar em formato \acs{PEM} (Base-64).

\subsection{Divergências}

\begin{enumerate}

\item Não foi possível implementar a parte servidora dessa extensão pois o OpenSSL não permite a configuração de um servidor com múltiplos certificados diferenciados apenas pelo emissor;

\item Como normalmente as aplicações que utilizam o OpenSSL especificam o
\emph{Certificate Store} coletivamente (via \verb|SSL_CTX_load_verify_locations|),
a API estendida obriga a adoção de um mesmo tipo de 
identificador para todos os certificados.

\end{enumerate}

\subsection{Testes de Conformidade}

Como os resultados são bastante semelhantes para os diversos tipos de IDs, 
apresenta-se aqui somente o resultado de teste efetuado com o tipo 
\verb|TLSX_CERT_ID_CERT_SHA1_HASH| com seguinte sintaxe na execução da aplicação 
\verb|s_client|:

\begin{lstlisting}
./s_client.sh -tlsx_trusted_ca_id cert_sha1_hash
\end{lstlisting}

O diretório de trabalho possuía somente um certificado cujo \emph{hash} \acs{SHA1} pôde 
ser obtido diretamente com o comando \verb|sha1sum|:

\begin{lstlisting}[caption={Cálculo do \emph{hash} \acs{SHA1} do certificado do CA}]
ebo@esparta:~/openssl-tlsx$ sha1sum 3ff23a0a.d 
|27399b0a3e46c936852cc3662b90d0e4d8635e78|  3ff23a0a.d
\end{lstlisting}

A equivalência do \emph{hash} obtido com esse comando e o enviado na mensagem 
\tlsHsCh abaixo (destacado em negrito e itálico no final da listagem), 
comprova a correção da implementação desta extensão.

\begin{lstlisting}[caption={Mensagem \tlsHsCh contendo a extensão \acs{TCI}}]
Secure Socket Layer 
  SSL Record Layer: Handshake Protocol: Client Hello 
    Content Type: Handshake (22) 
    Version: TLS 1.0 (0x0301) 
    Length: 72 
    Handshake Protocol: Client Hello 
      Handshake Type: Client Hello (1) 
      Length: 68 
      Version: TLS 1.0 (0x0301) 
      Random.gmt_unix_time: Jan  5, 2006 16:20:24.000000000 
      Random.bytes 
      Session ID Length: 0 
      Cipher Suites Length: 2 
      Cipher Suites (1 suite) 
	Cipher Suite: TLS_RSA_WITH_RC4_128_SHA (0x0005) 
      Compression Methods Length: 1 
      Compression Methods (1 method) 
	Compression Method: null (0) 
      Extensions Length: 25 
      Extension: trusted_ca_keys 
	Type: trusted_ca_keys (|0x0003|) 
	Length: |21| 
	Data (21 bytes) 
 
0000  00 00 00 00 00 00 00 00 00 00 00 00 08 00 45 00   ..............E. 
0010  00 75 22 ee 40 00 40 06 19 93 7f 00 00 01 7f 00   .u".@.@......... 
0020  00 01 c7 a9 01 bb eb ae f5 b1 eb 59 e3 a9 50 18   ...........Y..P. 
0030  7f ff 72 72 00 00 16 03 01 00 48 01 00 00 44 03   ..rr......H...D. 
0040  01 43 bd 63 68 29 5c 97 3f 14 37 d7 76 c0 11 35   .C.ch)\.?.7.v..5 
0050  6e 25 fc bd 6a d8 91 e9 1d b9 2e 85 cf 39 38 56   n%..j........98V 
0060  bb 00 00 02 00 05 01 00 00 19 |00 03 00 15 03 27|   ..........|.....'| 
0070  |39 9b 0a 3e 46 c9 36 85 2c c3 66 2b 90 d0 e4 d8|   |9..>F.6.,.f+....| 
0080  |63 5e 78|                                          |c^x|
\end{lstlisting}

