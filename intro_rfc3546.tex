%%%%%%%%%%%%%%%%%%%%%%%%%%%%%%%%%%%%%%%%%%%%%
%% Introdução a RFC 3546
%% Copyright 2006 Eliézio Batista de Oliveira
%%%%%%%%%%%%%%%%%%%%%%%%%%%%%%%%%%%%%%%%%%%%%

\chapter{As extensões do protocolo TLS}

As recomendações presentes na RFC 3546 podem ser desmembradas em duas 
partes distintas. A primeira trata da especificação de um mecanismo genérico 
de codificação e negociação das extensões ao TLS. A segunda parte especifica
6 extensões que, em geral, visam reduzir o \emph{overhead} do protocolo TLS.

As extensões codificadas devem ser acrescentadas ao final das mensagens 
\tlsHsCh e \tlsHsSh. Essa abordagem permite a compatibilidade com 
as implementações de TLS pré-existentes, uma vez que a sua 
especificação determina claramente que (\rfcurl{2246}{7.4.1.2}):
 
\begin{quote}
{\smaller
\begin{verbatim}
Forward compatibility note: 
    In the interests of forward compatibility, it is permitted for a 
    client hello message to include extra data after the compression 
    methods. This data must be included in the handshake hashes, but 
    must otherwise be ignored. This is the only handshake message for 
    which this is legal; for all other messages, the amount of data 
    in the message must match the description of the message 
    precisely.
\end{verbatim}
}
\end{quote}

As principais diretrizes na negociação das extensões são:

\begin{itemize}
\item Somente ao cliente é permitido propor extensões;
\item O servidor pode, para cada extensão proposta, tomar uma das seguintes ações:
\begin{enumerate}
    \item Aceitá-la e sinalizar essa aceitação acrescentando a mesma extensão
    ao \tlsHsSh;
    \item Ignorá-la silenciosamente;
    \item Rejeitá-la enviando uma mensagem \emph{Alert} para o cliente e
    abortando o \emph{handshake} imediatamente.
\end{enumerate}
\item Caso o servidor não confirme explicitamente a extensão proposta, o
cliente pode abortar a conexão caso julgue a extensão imprescindível.
\end{itemize}

A segunda parte da RFC~3546 propõe seis extensões específicas que acrescentam
novas funcionalidades ao protocolo TLS. Soma-se a elas uma sétima extensão
não-oficial especificada nesse documento e também implementada no OpenSSL.

Apresenta-se a seguir uma descrição sistemática de cada uma dessas extensões,
enfatizando os seguintes aspectos:

\begin{description}

\item[Referência] - Indica o trecho de documentação que contém os detalhes
técnicos sobre a extensão;

\item[Sinopse] - Um breve histórico da questão a ser resolvida e/ou aprimorada
pela extensão;

\item[Propósito da Extensão] - Descreve a solução proposta pela extensão;

\end{description}

\section{\acf{SNI}}

\begin{description}[\breaklabel]

\item[Referência]
	\rfcurl{3546}{3.1}.
	
\item[Sinopse]
	Servidores web, principalmente os comerciais, normalmente hospedam mais de um \emph{site}
	através de um mecanismo chamado \emph{name-based virtual hosting}, disponível em praticamente
	todos os servidores HTTP existentes. A seleção do domínio é feita pelo cliente HTTP (\emph{browser})
	através da campo `\verb|Host|' presente no \emph{request} HTTP \cite[seção~14.23]{rfc_http}.

	Nos servidores HTTPS a seleção do domínio precisa ser efetuada precocemente, pois é
	decisiva para a seleção do certificado do servidor a ser enviado, podendo também impactar na
	escolha de outros parâmetros da sessão.

	O campo `\verb|Host|' é, portanto, ineficiente para esta seleção uma vez que o \emph{request}
	HTTP é transmitido tardiamente, após a conclusão do \emph{handshake} TLS.

\item[Propósito da Extensão]
	A solução natural oferecida nesta extensão consiste na inclusão do nome do \emph{host} na primeira
	mensagem enviada pelo cliente, que é a \tlsHsCh.

\end{description}


\section{\acf{MFL}}

\begin{description}[\breaklabel]

\item[Referência]
	\rfcurl{3546}{3.2}.

\item[Sinopse]
	Na camada \emph{record layer}, cada registro recebido só pode ser disponibilizado
	para o protocolo superior após a verificação da sua integridade. Esse requisito
	obriga o cliente a dispor de memória suficiente para armazenar o maior registro 
	admitido pela especificação que é cerca de 16 kB.
	
\item[Propósito da Extensão]
	Essa extensão permite que o cliente estabeleça um limite superior para o tamanho de um 
	fragmento TLS, podendo variar de 512 até 4096 bytes.

\end{description}


\section{\acf{CCU}}

\begin{description}[\breaklabel]

\item[Referência]
	\rfcurl{3546}{3.3}.

\item[Sinopse]
	Sempre que a autenticação do cliente é solicitada pelo servidor, o cliente deve
	enviar uma mensagem \tlsHsC contendo um ou mais certificados.
	
\item[Propósito da Extensão]
	Com o uso dessa extensão, a mensagem \tlsHsC é substituída pela \tlsHsCu que
	conteria uma ou mais \acsp{URL} para a obtenção de toda a cadeia de certificados do 
	cliente.

\end{description}


\section{\acf{TCI}}

\begin{description}[\breaklabel]

\item[Referência]
	\rfcurl{3546}{3.4}.

\item[Sinopse]
	A princípio, um servidor pode estar configurado com mais de um certificado 
	para um mesmo \emph{Common Name} emitidos por \acsp{CA} diferentes.
	
\item[Propósito da Extensão]
	Essa extensão serviria então para informar ao servidor quais são os \acsp{CA} conhecidos e 
	confiáveis pelo cliente. O servidor teria então um critério para efetuar uma escolha 
	inequívoca sobre qual certificado X.509 deve empregado no \emph{handshake} TLS.

\end{description}


\section{\acf{TMAC}}

\begin{description}[\breaklabel]

\item[Referência]
	\rfcurl{3546}{3.5}.

\item[Sinopse]
	O MAC acrescentado a cada mensagem cifrada TLS, que pode ter 16 ou 20 bytes de tamanho
	conforme o algoritmo de \emph{Hash} em uso, pode representar uma parcela significativa
	nos dados trafegados, particularmente em comunicações interativas.

\item[Propósito da Extensão]
	Essa extensão permite que tanto cliente quanto servidor passem a usar um \acs{HMAC} reduzido 
	(truncado) a 10 bytes.

\end{description}


\section{\acf{CSR}}

\begin{description}[\breaklabel]

\item[Referência]
	\rfcurl{3546}{3.6}.

\item[Sinopse]
	O protocolo \ac{OCSP} \cite{rfc_ocsp} permite verificar em tempo real se um certificado foi 
	revogado ou não pelo \acs{CA} emissor.
	
\item[Propósito da Extensão]
	Com o uso dessa extensão, o servidor atuaria como um \emph{proxy} para a 
	obtenção de uma resposta \ac{OCSP} sobre o estado (revogado ou não-revogado) do 
	certificado do próprio servidor, dispensando assim o estabelecimento de uma segunda conexão TCP
	com o servidor (``\emph{responder}'') OCSP.
	
	O uso do \emph{proxy} seria benéfico também para os clientes que estivessem sujeitos a uma
	conectividade restrita, restrição comumente empregada em \acsp{VPN} para redes corporativas.

\end{description}


\section{\acf{SCCI}}

\begin{description}[\breaklabel]

\item[Referência]
	Draft em andamento. \\
	Autor: Eliézio Oliveira

\item[Sinopse]
	Como a cada novo \emph{handshake} TLS (portanto, sem \emph{``session resumption''}), o 
	servidor envia quase sempre a mesma seqüência de certificados, que 
	invariavelmente corresponde a maior parte do volume de dados trafegados, 
	decidimos acrescentar uma sétima extensão, a fim de evitar esse tráfego 
	redundante.

\item[Propósito da Extensão]
	Ao propor essa nova extensão, o cliente apresenta os identificadores de cada 
	certificado presente na cadeia de certificados previamente obtida. Esses 
	identificadores são exatamente os mesmos utilizados na extensão \acl{TCI}.

	Ao concordar com essa extensão, o servidor terá a opção de enviar uma 
	mensagem \tlsHsC vazia, mas ainda assim terá a alternativa de mandar a 
	cadeia inteira. O cliente deve, portanto, estar preparado para tratar os dois 
	casos ao processar a mensagem \tlsHsC enviada pelo servidor.

	Para eliminar qualquer possibilidade desta extensão deteriorar a segurança do 
	protocolo TLS, optou-se por incluir toda a seqüência de certificados no cômputo 
	do resumo criptográfico a ser publicado na mensagem \tlsHsF. Essa operação 
	deve ser efetuada imediatamente após a inclusão da mensagem \tlsHsC  
	recebida (cliente) ou enviada (servidor).

\end{description}

