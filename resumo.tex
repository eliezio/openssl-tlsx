%%%%%%%%%%%%%%%%%%%%%%%%%%%%%%%%%%%%%%%%%%%%%
%% Resumo
%% Copyright 2006 Eliézio Batista de Oliveira
%%%%%%%%%%%%%%%%%%%%%%%%%%%%%%%%%%%%%%%%%%%%%

\chapter*{Resumo}

\begin{center}

{\Large\bfseries \pfcTitulo}

\end{center}

O TLS, sucessor oficial do SSL, tem se consolidado como o protocolo de segurança
mais utilizado na Internet, estando embutido em praticamente todos os
navegadores e servidores HTTP.

A sua utilização em sistemas embutidos tem, entretanto, conflitado com os
limitados recursos que estes equipamentos dispõem. A RFC~3546
apresenta algumas extensões ao TLS visando minimizar o ônus da sua adoção.

Este texto descreve a implementação destas extensões realizadas pelo autor
no OpenSSL, um \emph{toolkit} TLS/SSL de código aberto largamente utilizado
em servidores HTTP.

Este texto apresenta também uma extensão adicional proposta e implementada 
pelo autor para reduzir ainda mais o volume de dados necessários para o
estabelecimento de uma conexão TLS.
