%%%%%%%%%%%%%%%%%%%%%%%%%%%%%%%%%%%%%%%%%%%%%
%% Introdução
%% Copyright 2006 Eliézio Batista de Oliveira
%%%%%%%%%%%%%%%%%%%%%%%%%%%%%%%%%%%%%%%%%%%%%
\unchapter{Introdução}

O protocolo \ac{TLS} \cite{rfc_tls}, previamente conhecido como \ac{SSL}, tem
se tornado o mecanismo mais popular para o estabelecimento de comunicações
seguras na Internet, sendo usado inclusive como alternativa ao \acs{IPsec} na
criação de \acp{VPN}.

Entretanto, a sua adoção em sistemas embutidos tais como telefones celulares e
\acsp{PDA} enfrenta alguns desafios:

\begin{itemize}
\item Relativa baixa capacidade de processamento desses dispositivos. Os 
diversos algoritmos criptográficos usados no TLS demandam, por 
natureza, uma alta utilização de CPU;
\item Sua relativa baixa capacidade de armazenamento, que pode limitar a sua
capacidade de autenticação;
\item Usam em geral meios de comunicação com banda limitada, especialmente em redes
compartilhadas onde o tráfego de dados não é prioritário em relação às demais mídias,
como é típico na rede \acs{GPRS};
\item A alta latência inerente às redes via satélite;
\item Algumas redes \emph{wireless} são comumente tarifadas por volume de dados.
\end{itemize}

Visando torná-lo mais leve para equipamentos sujeitos a essas restrições, o
\emph{IETF TLS Working Group} publicou em Junho de 2003 a RFC~3546
\cite{rfc_tlsext}, intitulado \emph{``Transport Layer Security (TLS) Extensions''},
que propõe diversas extensões ao TLS, preservando a
retrocompatibilidade com as versões pré-existentes do protocolo.

As reduções promovidas por essas extensões oficiais, entretanto, não tratam da redução da mensagem
que contém o certificado do servidor, responsável por aproximadamente 70\% do \emph{overhead}
do TLS no volume de dados durante o estabelecimento de uma conexão.

Face a esta lacuna, uma extensão adicional foi então concebida por este autor para tratar especificamente
desta notável redução, aplicável a uma grande parte das conexões TLS.

O objetivo desse projeto final de curso é implementar esta extensão adicional e a maioria das
extensões oficiais da RFC~3546 na pilha TLS/SSL de código aberto OpenSSL.
