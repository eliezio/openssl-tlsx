%%%%%%%%%%%%%%%%%%%%%%%%%%%%%%%%%%%%%%%%%%%%%
%% Abstract
%% Copyright 2006 Eliézio Batista de Oliveira
%%%%%%%%%%%%%%%%%%%%%%%%%%%%%%%%%%%%%%%%%%%%%

\chapter*{Abstract}

\begin{center}

{\Large\bfseries \pfcTituloEN}

\end{center}

TLS, the official successor of SSL, is becoming the mostly widely used security protocol on data communication networks, notably the Internet, virtually available on all web browsers and servers.

Its usage by embedded systems faces, however, some challenges due to the very restricted resources available to those systems. The IETF's RFC 3546 recommends some extensions especially targeted to minimize the protocol overhead.

This dissertation describes the implementation of these extensions made by this author by means of modifications on OpenSSL, an open-source TLS/SSL toolkit commonly employed in HTTP servers.

Furthermore, this text presents an additional extension advised and implemented by this author alongside the official extensions, aimed to reduce the volume of network traffic during the TLS connection establishment.
